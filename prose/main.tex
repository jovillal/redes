\documentclass[12pt]{article}
\usepackage[utf8]{inputenc}
\usepackage[english]{babel}
\usepackage{amsmath}
\usepackage{amsfonts}
\usepackage{amssymb}
\usepackage{amsmath}
\usepackage{amsthm}
\usepackage{latexsym}
\usepackage{amscd}
\usepackage{graphics}
\usepackage{graphicx}
\usepackage[pdftex,linktocpage=true,bookmarks=true,bookmarksnumbered=true,colorlinks,linkcolor=blue,citecolor=red]{hyperref}
%\usepackage{anysize}
%\usepackage{epstopdf}
%\usepackage{pst-all}
%\usepackage{pstricks}
%\usepackage{url}
%\usepackage{float}
%\usepackage[T1]{fontenc}
%\usepackage{textcomp}
%\usepackage{color}

%%%%%%%%%%%%%%%%%%%%%%

%opening
\title{A model of packages moving through a network.}
\author{J. Villalobos}

\date{\today}

%%%%%%%%%%%%%%%%%%%%%%

\begin{document}

\maketitle

\begin{abstract}
  \textcolor{red}{\large {\bf FALTA}}
\end{abstract}


%%%%%%%%%%%%%%%%%%%%%%%%%%%%%%%
%\section{Introduction}
%%%%%%%%%%%%%%%%%%%%%%%%%%%%%%%

%%%%%%%%%%%%%%%%%%%%%%%%%%%%%%%
\section{The Model}
%%%%%%%%%%%%%%%%%%%%%%%%%%%%%%%

We simulate traffic on a network using the following setup:
\begin{itemize}
  \item Time is discrete. The unit is a tick; we represent it by $t$.
  \item A graph represents the network; we use $M$ for the number of nodes.
  \item We use packages to represent traffic that may move between nodes each tick. 
  \item There are $P$ packages within the network at any given tick. $P$ is kept constant through a simulation.
  \item We use $K$ for the maximum number of packages that a node may hold. 
    Therefore, each node $m \in [0,1,\ldots,M]$ has a number of packages $0 \le p_m(t) \le K$ at a given time.    
  \item Since the total number of packages is kept constant through the simulation, we have that $\sum_{m=1}^M p_m(t) = P$, for all $t$.
  \item Individual nodes operate under the following rules:
  \begin{itemize}
    \item Each has a fixed capacity of $K$ (the maximum number of packages that can occupy a node at a given tick).
    \item Each tick, every node chooses a neighbor node at random and tries to send a package to it.
    \item Each tick a node may receive any number of packages between 0 and its in-degree.
    \item If the number of received packages plus the actual number of packages is larger than $K$, no packages are accepted (they are all returned to the neighbors that sent the packages).
  \end{itemize}
  \item Each edge has one of two states: open (green) or closed (red).
  \begin{itemize}
    \item Open means that a package may travel between the nodes connected by the edge; closed means that it may not.
    \item The state of edges changes with time based on two parameters: $\omega_{j,k} \ge 0$ and $\phi_{j,k} \in [0, 2 \pi]$.
      The subscripts $j$, $k$ indicate that the edge connects node $j$ with node $k$.
    \item An edge that connects node $j$ to node $k$ is open at time $t$ if $\sin(\omega_{j,k} t + \phi_{j,k}) \ge 0$.
    \item An edge that connects node $j$ to node $k$ is closed at time $t$ if $\sin(\omega_{j,k} t + \phi_{j,k}) < 0$.
  \end{itemize}
\end{itemize}



%%%%%%%%%%%%%%%%%%%%%%%%%%%%%%%
\section{Some experiments}
%%%%%%%%%%%%%%%%%%%%%%%%%%%%%%%

We will use four different networks on which to run simulations.
A set of connections defines a graph.
We use the notation $u \leftrightarrow v$ to denote a two-way or undirected link between node $u$ and $v$ (a single arrow indicates a directed link).
The definitions of the four networks are:
\begin{enumerate}
  \item A complete network is one on which all possible connections are available.
    This is $\{u \leftrightarrow v\}$, $\forall u \neq v$.
  \item The second network is a circle network, one on which every node has two connections.
    The graph is defined by: $\{1  \leftrightarrow 2, 2 \leftrightarrow 3, 3 \leftrightarrow 4, \dots m \leftrightarrow m+1, \ldots, M-1 \leftrightarrow M, M \leftrightarrow 1\}$.
  \item A square grid, a network in which every node has four neighbors (we call these the inner nodes), except for four nodes (corners) with precisely two neighbors, and some nodes (the edges) that have three neighbors each.
    We will only use square grids; this means that $M$ will always be an integer square.
  \item A Barabasi-Albert graph, constructed starting from a circular graph of three nodes, subsequently add a vertex with two edges at each step. The edges are attached to vertices at random, following a distribution proportional to the vertex degree.
    By following this algorithm, the typical network obtained is one where there is a hub node (one with a very high degree), and most nodes have a small number of neighbors.
\end{enumerate}

We propose two different setups for simulations: keeping the edges open at all time and opening and closing the edges according to fixed values of $\omega_{j,k} \ge 0$ and $\phi_{j,k}$.
We start running simulations on undirected networks and follow with directed networks.

\subsection{Undirected Networks}
\label{sec:undirected}

We build average flow, $\bar{J}$, versus density, $\rho$, plots.
The average flow is the mean number of packages that moved on an ensemble of simulations (we run 40 simulations of 300 ticks each). 
We normalize the flow to the number of nodes on the network.
The density $\rho$ is the number of packages in the network divided by the product of the number of nodes and the capacity of the nodes; this is $\rho = \frac{P}{M \times K}$.

On an undirected network, if a package can travel from node $a$ to node $b$, then it can always do it from $b$ to $a$.


\subsubsection{All links open}
\label{sec:undirected-open}

We start by running simulations on undirected networks of $M=100$ nodes with open edges.
All simulations are carried out with all the links open all the time. 

We get that $\bar{J}$ behaves as depicted in Fig. \ref{fig:JvsrhoAllOpen}.
Note that for the four types of networks and different values of $K$ (2, 5, 10, and 100), the average flow $\bar{J}$ starts at 0, for $\rho=0$ and grows as $\rho$ grows, then it comes to a plateau and starts decreasing in an almost symmetric way.
The maximum height depends on the value of $K$, and the kind of network determines its position.

\begin{figure}[!hbt]
  \centering
  \includegraphics[width=\textwidth]{plots/fundamentalUndirected.pdf}
  \caption{Average flow $\bar{J}$ Vs. density $\rho$ for four types of networks.}
  \label{fig:JvsrhoAllOpen}
\end{figure}

We expect that the flow is 0 for both $\rho = 0$ and $\rho =1$ since in the first case we have $m=0$ (no packages in the network), and for $\rho = 1$ we have $m = M K$, which means that all nodes have $K$ packages (no movement is possible in this situation).
The flow for the scale-free network is always inferior to the flow for other networks; something expected given that the degree distribution of such network is different from the other ones.
Note that the degree distribution is uniform for the complete and circle networks, and it is almost uniform for the grid.
On a scale-free graph, there are few nodes with lots of neighbors; while most modes have a small degree, the hub fills up with packages and hinders flow.
Note that this is the case even for high values of $K$

\begin{figure}[!hbt]
  \centering
  \includegraphics[width=\textwidth]{plots/ocupationVarianceComplet.pdf}
  \caption{Ocupation percentage Vs. density $\rho$ for the coplete network for $K=2$ (left), and $k=100$ (right). 
    The ocupation is shown for three nodes: dmin, dmax and davg (see text for the definitions).}
  \label{fig:perOcComplete}
\end{figure}

Figures \ref{fig:perOcComplete} through \ref{fig:perOcScaleFree} show the ocupation percentage Vs. $\rho$ for the different networks and different values of $K$.
We use one simulation of 300 ticks for 40 different values of $\rho$.
The occupation percentage is the mean number of packages at a given node, divided by $K$ (the capacity of the node); we calculate this percentage as the mean of the packages that are on the node on a given simulation, we plot this mean with bars that mark standard deviation normalized to the number of ticks used on the simulation.
Each plot shows the behavior of a different node picked randomly based on the following definitions:
\begin{itemize}
\item dmin refers to a node that has the minimum degree in the network.
\item dmax is a node that has the maximum degree of the network.
\item davg is a node, as close as possible, to the floor of the network's average degree.
\end{itemize}

Figure \ref{fig:perOcComplete} shows the ocupation percentage for the complete network. 
We randomly pick dmin, dmax, and davg since all nodes have the same degree (the number of nodes minus one).
The subplot on the left was made with maximum occupation $K=2$, the one on the right with $K=100$.
As expected, when there are no packages in the network ($\rho =0$), the occupation is 0, and when the number of packages is $K$ in every node ($\rho=1$), the occupation is 1.
When the maximum occupation number is small ($K=2$), the occupation correlates linearly with the density of packages; we can say it is uniform through the network; this is because there cannot be much flow due to the restriction of the number of packages per node.
When the maximum occupation number is large ($K=100$), the occupation behaves erratically, and all three nodes behave differently; unless $\rho$ is close to 0 or to 1, the occupation on each node is different from the one in another.

\begin{figure}[!hbt]
  \centering
  \includegraphics[width=\textwidth]{plots/ocupationVarianceCircle.pdf}
  \caption{Ocupation percentage Vs. density $\rho$ for the circle network for $K=2$ (left), and $k=100$ (right). 
    The ocupation is shown for three nodes: dmin, dmax and davg (see text for the definitions).}
  \label{fig:perOcCircle}
\end{figure}

Figure \ref{fig:perOcCircle} shows the occupation percentage for the circle network. 
Since all the nodes have the same degree (they all have two neighbors), we randomly pick dmin, dmax, and davg.
The subplot on the left was made with maximum occupation $K=2$, the one on the right with $K=100$.
Given that this network is similar to the complete one (all nodes have the same degree), the observed behavior is very similar. 
When the maximum occupation number is small ($K=2$), the occupation correlates linearly with the density of packages, we can say it is uniform through the network; there cannot be much flow due to the restriction in the number of packages per node added to the restriction that there are few nodes that may receive packages.
When the maximum occupation number is large ($K=100$), the occupation behaves erratically, and all three nodes behave differently; unless $\rho$ is close to 0 or to 1, the occupation on each node is different from the one in another.
As with the complete network, this is because all nodes have the same degree.

\begin{figure}[!hbt]
  \centering
  \includegraphics[width=\textwidth]{plots/ocupationVarianceGrid.pdf}
  \caption{Ocupation percentage Vs. density $\rho$ for the grid network for $K=2$ (left), and $k=100$ (right). 
    The ocupation is shown for three nodes: dmin, dmax and davg (see text for the definitions).}
  \label{fig:perOcGrid}
\end{figure}

Figure \ref{fig:perOcGrid} shows the occupation percentage for the grid network. 
On this networks, we have three sets of nodes (classified by their degree); we pick dmin as one of the corner nodes (2 neighbors), davg as one of the edge nodes (3 neighbors), and dmax as one of the inner nodes (4 neighbors).
The subplot on the left was made with maximum occupation $K=2$, the one on the right with $K=100$.
We observe a different behavior when compared to the previous networks. 
When the maximum occupation number is small ($K=2$), the occupation correlates linearly with the density of packages, except for the corner node that starts with a slope slightly less than 1, and after crossing the $\rho > 0.5$ value, changes it to a value bigger than 1. 
When the maximum occupation number is large ($K=100$), the occupation becomes more predictable than for the previous networks; the corners and edges have low values of occupancy for most of the density values and only start getting occupied after a high value of $\rho$; the inner nodes have a high slope for small values of $\rho$ and saturate to high occupancy near $\rho \gtrsim 0.5$.

\begin{figure}[!hbt]
  \centering
  \includegraphics[width=\textwidth]{plots/ocupationVarianceScaleFree.pdf}
  \caption{Ocupation percentage Vs. density $\rho$ for the scale free network for $K=2$ (left), and $k=100$ (right). 
    The ocupation is shown for three nodes: dmin, dmax and davg (see text for the definitions).}
  \label{fig:perOcScaleFree}
\end{figure}

Figure \ref{fig:perOcScaleFree} shows the ocupation percentage for the scale free network. 
On these networks, the nodes have different degrees. 
However, most nodes have a small number of neighbors, while few nodes have a high degree; we show this in Fig. \ref{fig:degreeDistSF}, using a histogram of the degree distribution for the network used in the simulations of this section; as can be seen, we have one node with 27 neighbors and more than 40 nodes with only two.
Based on the distribution, we pick dmax as the node with the highest degree, dmin as one of the nodes with two neighbors, and davg as one of the nodes with four neighbors.
The subplot on the left was made with maximum occupation $K=2$, the one on the right with $K=100$.
We observe a different behavior, especially for the hub node (dmax).
When the maximum occupation number is small ($K=2$), the occupation correlates almost linearly with the density of packages, for both davg and dmin; the hub has a very pronounced slope that quickly saturates and then decreases, almost in an exponential form, to finally rise again to meet the requirement of occupation for $\rho =1$.
When the maximum occupation number is large ($K=100$), the occupation becomes ordered, as with the grid network, the dmin nodes have a very low occupation and start growing after $\rho > 0.5$; the nodes with a slightly higher degree are occupied for small values of the density and saturate quickly.
The hub saturates to near one occupancy very quickly and maintains its occupancy number for most values of $\rho$.

\begin{figure}[!hbt]
  \centering
  \includegraphics[width=0.75\textwidth]{plots/degDistributionScaleFree.pdf}
  \caption{Histogram for the degree of the scale free network, as expected there is a hub with 27 neighbours while most of the nodes have less than 5 neighbours.}
  \label{fig:degreeDistSF}
\end{figure}


\subsubsection{Links changing states}
\label{sec:undirected-red-green}


	
\subsection{Directed Networks}
\label{sec:directed}

Under construction

\end{document}